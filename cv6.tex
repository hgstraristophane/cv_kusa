\documentclass[12pt]{letter}
\usepackage[top=0.1cm, bottom=0cm, left=0.1cm, right=0.1cm]{geometry} % Marges réduites haut/bas

\usepackage[utf8]{inputenc}
\usepackage[T1]{fontenc}
\usepackage[french]{babel}
\usepackage{geometry}


\usepackage{multicol}
\usepackage{lipsum} % Pour générer du texte d'exemple

\usepackage{xcolor}
\usepackage{framed}

\definecolor{shadecolor}{rgb}{10,0.9,0.9} % couleur du fond

\usepackage{tcolorbox}


\definecolor{couleurB1}{rgb}{0,0.2,0.9} % couleur du fond

\usepackage{graphicx}
\usepackage{tikz}



%\signature{KUETE SONGMENE}
%\address{KUETE SONGMENE\\
%TEL: +237 654 41 61 07 / +237 652 84 16 18\\
%Email: kuetearistophane@gmail.com}

%\date{Dschang, le 25 octobre 2024}


\begin{document}
	
\pagestyle{empty}%supprime la numreotation



\begin{letter}{}
		\pagestyle{empty}%supprime la numreotation
		\noindent
		\begin{minipage}[t]{0.28\textwidth}
			%\textbf{Bloc de gauche}\\
					\begin{tikzpicture}
					\node[draw=white, rounded corners=10pt, line width=0.9pt] 
					{\includegraphics[width=1\linewidth]{image6.jpg}};
					\end{tikzpicture}
					\begin{tcolorbox}[colback=blue!20, colframe=couleurB1!80!black, title=AUTRE]
										
										\textbf{KUETE SONGMENE ARISTOPHANE}\\
											25 ans\\
											 Né le : 21/10/2000 à Balatchi\\
											célibataire
											Pas d’enfant  .
											\textbf{Tél /whatSapp :}\\
											+237 654 41 61 07\\
											\textbf{e-mail :}\\
											kuetearistophane@gmail.com\\\\
											
											\textbf{Références :}
										\textbf{Pr Talla Mbe Jimmi} (UDS, departement de physique, domaine de recherche système dynamique, chaos et optoélectronique)\\
											\textbf{Tél /whatSapp :}
											+237 6 51 38 60 02\\
											\textbf{E-mail :}\\ jhtallam@yahoo.fr\\ \\
											
											\textbf{Dr. TCHAMADA André Rodrigue} (UDS, Fasa-genie rurale, Directeur de mémoire, domaine de recherche application des méthodes de l’IA  à l’agriculture et télémédecine)\\
											\textbf{Tél /whatSapp :}\\
											+237 6 77 77 58 24 78\\
										\textbf{E-mail:}\\ rodriguetchamda@yahoo.fr
										
										
											
											
											\end{tcolorbox}

	
			
		\end{minipage}\hfill %alignement horizontal
		\begin{minipage}[t]{0.71\textwidth}
	
				
				\vspace*{-10em}
		\hspace{2em}	\textbf{\LARGE{\textcolor{blue}{KUETE SONGMENE ARISTOPHANE}}}
		\vspace*{0.3em}
		
		\hspace{2em}	\textbf{\LARGE{\textcolor{blue}{MASTER EN PHYSIQUE}}}
				\vspace*{7em}
				
			
			\begin{tcolorbox}[colback=white!40, colframe=couleurB1!80!black, title=PROFIL]
				\textbf{Enseignant de mathématique et de physique expérimenté}, titulaire d’un \textbf{Master en physique option Électronique, Électrotechnique et Automatique}. Thème : Classification des œufs à couver à l’aide du Raspberry pi et du Deep Learning. Je possède une solide expertise dans l’enseignement de l’informatique, de la\textbf{ programmation en générale} et de la conception des \textbf{ applications mobile} androïde, des sites webs et des applications web, Ainsi que la conception de projets électroniques.\textbf{ Je suis également un excellent communicant, capable d’expliquer, de transmettre des connaissances complexes} de manière claire et concise. Mon expérience en travail d’équipe m’a permis de développer des compétences en leadership, en gestion de projet et en communication efficace. Je suis dynamique, motivé et passionné par l’enseignement et la recherche. 
			\end{tcolorbox}
			
					\begin{tcolorbox}[colback=white!40, colframe=couleurB1!80!black, title=EDUCATION]
							
							
							\begin{itemize}
								\item  \textbf{\textcolor{blue}{ 2023 Master en Physique (EEA)}} Université de Dschang (Dschang, Cameroun)
								\item  \textbf{\textcolor{blue}{2021 Licence en Physique(EEA) }} Université de Dschang (Dschang, Cameroun)
								\item \textbf{\textcolor{blue}{ 2018 Baccalauréat C}}  Lycée bilingue de balatchi (Bamboutos, Cameroun)
								
							\end{itemize}
						\end{tcolorbox}
						
									\begin{tcolorbox}[colback=white!40, colframe=couleurB1!80!black, title=COMPETANCES]
										
										
										\begin{itemize}
											\item \textbf{\textcolor{blue}{Langues :} } Anglais et Français : Parler, écrire.
											\item \textbf{\textcolor{blue}{Langages de Programmation :}}  Python, Java, HTML/CSS , JavaScript, SQL, PHP, C,  Matlab,  Arduino,  Android, VBA, Latex, VHDL, …
											\item \textbf{\textcolor{blue}{Logiciels Informatiques:}} Jupiter, Spider, Matlab, Arduino, Proteus, Eclipse, Excel, Android Studio, VISUAL STUDIO CODE,  Excel, Word, PowerPoint, TeXstudio, MathType . . . 
											
											\item \textbf{\textcolor{blue}{ Technique :}} Installation électrique bâtiment, Réalisation des projets électroniques, développement d’application mobile et web.
											
											
										\end{itemize}
									\end{tcolorbox}
									
									
										\begin{tcolorbox}[colback=white!40, colframe=couleurB1!80!black, title=EXPERIENCE DE TRAVAIL]
											
											
											
											
											
											
											
											
											\begin{itemize}
											
											\item \textbf{\textcolor{blue}{2021-Présent	}}\\
										    \textbf{Fonction :} Enseignant Établissements :\\
											 •	Lycée Bilingue de Latsuet-Tsinmeliet\\
											 •	Enseignant de physique et de PCT\\
											 •	Enseignant d’informatique\\
										     •	Personne de soutien académique en informatiques,
											
											 physiques, mathématiques, circuits électriques.
												•	Animateur pédagogique indépendant
												
											\item \textbf{\textcolor{blue}{	 2020-2021}}\\
												Fonction : Assistant aux travaux dirigés et travaux pratiques Établissement : Université de Dschang, Ville : Dschang, Responsabilités :\\
												•	Assistant aux travaux dirigés de circuit linéaire niveau 1\\
												•	Assistant aux travaux pratiques de physique niveau 1\\
												
											\item \textbf{\textcolor{blue}{	2024}}\\  Développement de l’application mobile Android MathEduc : permettant aux lycéens de mieux résoudre certains problèmes mathématiques les plus fréquents.
												
									
												
											\end{itemize}
												
										\end{tcolorbox}
			
			
			
			
		\end{minipage}
			\noindent
			\begin{minipage}[t]{0.28\textwidth}
				
			\end{minipage}
				
		\hfill %alignement horizontal
		\begin{minipage}[t]{0.71\textwidth}
			
			\begin{tcolorbox}[colback=white!40, colframe=couleurB1!80!black, title=EXPERIENCE DE TRAVAIL(suite)]
				
				
				
				
				
				
				
				
				\begin{itemize}
					
					\item \textbf{\textcolor{blue}{2021-Présent	}}\\
					\textbf{Fonction :} Enseignant Établissements :\\
					•	Lycée Bilingue de Latsuet-Tsinmeliet\\
					•	Enseignant de physique et de PCT\\
					•	Enseignant d’informatique\\
					•	Personne de soutien académique en informatiques,
					
					physiques, mathématiques, circuits électriques.
					•	Animateur pédagogique indépendant
					
					\item \textbf{\textcolor{blue}{	 2020-2021}}\\
					Fonction : Assistant aux travaux dirigés et travaux pratiques Établissement : Université de Dschang, Ville : Dschang, Responsabilités :\\
					•	Assistant aux travaux dirigés de circuit linéaire niveau 1\\
					•	Assistant aux travaux pratiques de physique niveau 1\\
					
					\item \textbf{\textcolor{blue}{	2024}}\\  Développement de l’application mobile Android MathEduc : permettant aux lycéens de mieux résoudre certains problèmes mathématiques les plus fréquents.
					
					\item \textbf{\textcolor{blue}{		Fin 2024 }}\\ Développement de l’application mobile Android MathEXO : permettant aux utilisateurs de répondre aux questions de mathématiques sous forme de QCM.
					
					\item \textbf{\textcolor{blue}{2025}} \\ Intégration des anciens sujets des concours IDE et de faculté de médecine dans les plateformes MathEXO et Bords Numériques (mathématiques et biologie), sous forme de QCM interactifs. Développement du site web dédié à la présentation et à l’accès à ces applications.
					
				\end{itemize}
			\end{tcolorbox}
			
			
			
		\end{minipage}	
			





	


\newpage

\end{letter}

\end{document}
